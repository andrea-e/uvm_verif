%%%%%%%%%%%%%%%%%%%%%%%%%%%%%%%%%%%%%%%%%
%
% Funkcionalna verifikacija hardvera
% 
%%%%%%%%%%%%%%%%%%%%%%%%%%%%%%%%%%%%%%%%%

Ova vežba se bavi nizovima u SystemVerilog jeziku, načinima kastovanja kao i
objektno-orijentisanim aspektima jezika. Objašnjeni su osnovni pojmovi u OOP
terminologiji, konvencije pri korišćenju i dati su primeri upotrebe. Ukratko su
opisane interfejs i \emph{package} konstrukcije.

%========================================================================================
% Section
%========================================================================================

\section{Nizovi}

U ovom poglavlju su objašnjeni nizovi u SystemVerilog jeziku. Dat je pregled
osnovnih vrsta i načina korišćenja, objašnjena razlika između \emph{packed} i
\emph{unpacked} nizova i objašnjen način upotrebe ugrađenih metoda. Dat je i
pregled redova (engl. \emph{queue}) u SystemVerilog-u.

%----------------------------------------------------------------------------------------

\subsection{Nizovi fiksne veličine}

Nizovi fiksne veličine se deklarišu navođenjem gornjeg i donjeg limita (po uzoru
na Verilog) ili navođenjem veličine (po uzoru na C). Sledeće dve deklaracije su
ekvivalentne i predstavljaju niz od 8 int promenljivih.

\begin{lstlisting}
int example [0 : 7];
int example2 [8];
\end{lstlisting}

Multidimenzioni nizovi se isto deklarišu navođenjem dimenzija nakon imena niza, npr:

\begin{lstlisting}
int example3 [8] [15];
\end{lstlisting}

Niz se može inicijalizovati koristeći sintaksu ispod:

\begin{lstlisting}
bit example4 [7:0] = '{0,0,0,0,0,0,0,1};
\end{lstlisting}

Čest način manipulisanja listama je \emph{for}, odnosno \emph{foreach} petlja.
Prilikom iteracije kroz niz koristeći \emph{for} petlju, pogodna je funkcija
\emph{\(\$\)size()} koja vraća veličinu niza.
\emph{Foreach} petlja, sa druge strane, ovo radi automatski.
Prilikom korišćenja \emph{foreach} petlje navede se niz i index, i zatim se vrši
iteracija kroz sve elemente niza.
Npr. sledeći primeri dupliraju vrednost svih članova niza:

\begin{lstlisting}
for(int i = 0; i < $size(example); i++) begin
  example[i] *= 2;
end

foreach(example[i])
  example[i] *= 2;
\end{lstlisting}
% $ komentar jer jedan znak sve posle pogresno oboji

\emph{Foreach} petlja za multidimenzionu listu ima malo drugačiju sintaksu.
Umesto navođenja dimenzija u odvojenim zagradama, dimenzije se navode u istoj
uglastoj zagradi odvojene zarezima. Npr:

\begin{lstlisting}
foreach (example3 [i, j]) example3[i][j] *= 2;
\end{lstlisting}

U nastavku je dato još par primera upotrebe nizova kako bi se bolje upoznali sa sintaksom.

\lstinputlisting[caption=nizovi fiksne velicine, label=lst:fixed_size_arrays]{code/v2_fixed_size_arrays.sv}

%----------------------------------------------------------------------------------------

\subsection{\emph{Packed} i \emph{unpacked} nizovi}

Ako se prisetimo prve vežbe i deklaracije željene veličine promenljivih, možemo
uočiti razliku između deklaracije promenljive željene veličine sa nizom fiksne
veličine, objašnjenim u prethodnom poglavlju. Zapravo obe ove deklaracije
predstavljaju niz podataka, ali dve različite vrste – \emph{packed} i
\emph{unpacked}. Na primeru ispod možemo videti razliku u sintaksi.

\begin{lstlisting}
bit [7 : 0] x; // packed niz
bit x [8]; // unpacked niz
\end{lstlisting}

Za neke tipove podataka je pogodno pristupati celom podatku, ali i podeliti ga u
manje elemente kojima se može pojedinačno pristupati (npr. pristupati bajtu
ođednom ili svakom bitu pojedinačno).
\emph{Packed} niz ovo omogućava.
On se može tretirati i kao niz i kao jedna vrednost.
Za razliku od \emph{unpacked} niza, dimenzije packed niza se navode pre imena,
kao deo tipa promenljive.
Takođe, obavezno je navesti dimenzije u formatu [MSB : LSB] (nije moguće
navesti samo veličinu).
\emph{Packed} nizovima se može pristupati kao običnoj promenljivoj.
Može im se dodeljivati vrednost, koristiti se u raznim izrazima i sl., ali se
može pristupati i pojedinačnim elementima, npr:

\begin{lstlisting}
logic [3 : 0] x, y = 0;
x = 4'hA;
y = x + 4'h8;
x[2] = y[0];
y[3 : 1] = 3'b010;
\end{lstlisting}

Glavna razlika između \emph{packed} i \emph{unpacked} nizova je u načinu čuvanja
vrednosti u memoriji.
\emph{Packed} niz se čuva u kontinualnom nizu bita, bez praznih mesta, što ne
mora biti slučaj sa \emph{unpacked} nizom.
Sledeće tabele ilustruju način smeštanja nizova u memoriji, ali i mogućnost
kombinovanja \emph{packed} i \emph{unpacked} dimenzija:

\begin{center}
  \resizebox{\textwidth}{!}{
  \begin{tabular}{|c|c|c|c|c|c|c|c|c|c|c|c|c|c|c|c|c|c|c|c|c|c|c|c|c|c|c|c|c|c|c|c|c|c|c|c|c|c|c|c|}
    \multicolumn{40}{c}{bit [3:0] [7:0] packed\(\_\)example}\\
    \hline
    &&&&&&&&7&6&5&4&3&2&1&0&7&6&5&4&3&2&1&0&7&6&5&4&3&2&1&0&7&6&5&4&3&2&1&0\\
    \hline
    &&&&&&&&\multicolumn{8}{ c| }{packed\(\_\)example[3]}&\multicolumn{8}{ c| }{packed\(\_\)example[2]}&\multicolumn{8}{ c| }{packed\(\_\)example[1]}&\multicolumn{8}{ c| }{packed\(\_\)example[0]} \\
    \hline
  \end{tabular}}
\end{center}

\begin{center}
  \begin{tabular}{|c|c|c|c|c|c|c|c|c|c|c}
    \multicolumn{11}{c}{bit [7:0] unpacked\(\_\)example [4]}\\\cline{0-9}
    &&7&6&5&4&3&2&1&0&unpacked\(\_\)example[0]\\\cline{0-9}
    &&7&6&5&4&3&2&1&0&unpacked\(\_\)example[1]\\\cline{0-9}
    &&7&6&5&4&3&2&1&0&unpacked\(\_\)example[2]\\\cline{0-9}
    &&7&6&5&4&3&2&1&0&unpacked\(\_\)example[3]\\\cline{0-9}
  \end{tabular}
\end{center}

\begin{center}
  \resizebox{\textwidth}{!}{
  \begin{tabular}{|c|c|c|c|c|c|c|c|c|c|c|c|c|c|c|c|c|c|c|c|c|c|c|c|c|c|c|c|c|c|c|c}
    \multicolumn{32}{c}{bit [2:0] [7:0] mix\(\_\)example [4]}\\
    \multicolumn{32}{c}{dat je i primer pristupanja određenim elementima} \\
    \multicolumn{6}{c}{mix\(\_\)example[0][2][5]} \\\cline{0-23}
    7&6&\cellcolor{gray!45}5&4&3&2&1&0&7&6&5&4&3&2&1&0&7&6&5&4&3&2&1&0&\multicolumn{8}{c}{mix\(\_\)example[0]} \\\cline{0-23}
    7&6&5&4&3&2&1&0&7&6&5&4&3&2&1&0&7&6&5&4&3&2&1&0&\multicolumn{8}{c}{mix\(\_\)example[1]} \\\cline{0-23}
    7&6&5&4&3&2&1&0&7&6&5&4&3&2&1&0&7&6&5&4&3&2&1&0&\multicolumn{8}{c}{mix\(\_\)example[2]} \\\cline{0-23}
    7&6&5&4&3&2&1&0&\cellcolor{gray!45}7&\cellcolor{gray!45}6&\cellcolor{gray!45}5&\cellcolor{gray!45}4&\cellcolor{gray!45}3&\cellcolor{gray!45}2&\cellcolor{gray!45}1&\cellcolor{gray!45}0&7&6&5&4&3&2&1&0&\multicolumn{8}{c}{mix\(\_\)example[3]} \\\cline{0-23}
    \multicolumn{8}{c}{ }&\multicolumn{6}{c}{mix\(\_\)example[3][1]} \\
    \end{tabular}
    }
\end{center}

%----------------------------------------------------------------------------------------

\subsection{Dinamički nizovi}

Dinamički niz je jednodimenzioni, \emph{unpacked} niz čija se veličina može
menjati tokom simulacije, odnosno ne mora biti poznata u vreme kompajliranja.
Deklariše se korišćenjem praznih uglastih zagrada [].
Dinamički niz ne zauzima prostor u memoriji do god se on eksplicitno ne kreira
korišćenjem metode \emph{new}[\textless veličina\textgreater].
Pored ove metode, često se koriste i metode \emph{size()} koja vraća trenutnu
veličinu niza, i metoda \emph{delete()} koja briše niz (rezultat je prazan niz).
U nastavku je dat primer korišćenja dinamičkih nizova.

\lstinputlisting[caption=Dinamicki nizovi, label=lst:dyn_array]{code/v2_dyn_array.sv}

%----------------------------------------------------------------------------------------

\subsection{Asocijativni nizovi}

Ukoliko je veličina niza nepoznata ili je redak prostor podataka (\emph{sparse
  data space}), dobro rešenje pružaju asocijativni nizovi.
Ovi nizovi čuvaju elemente u \emph{sparse} matrici i zauzimaju memoriju tek kada
se koristi element.
Takođe, indeks niza ne mora biti celobrojni, već može biti bilo kog tipa.
Asocijativni niz implementira \emph{lookup} tabelu željenog tipa, gde indeks
predstavlja \emph{lookup} ključ.
Primer upotrebe asocijativih nizova bila bi, npr., verifikacija memorije velikog
opsega.
Tokom simulacije se možda vrši pristup na samo nekoliko adresa i u asocijativnom
nizu se čuvaju vrednosti tih lokacija, umesto čitave memorije što znatno štedi
na utrošenom memorijskom prostoru.\\

Sintaksa deklaracije asocijativnih nizova je:

\begin{lstlisting}
<tip_podatka> <id_niza> [<tip_indeksa>];
\end{lstlisting}

gde \textless tip\(\_\)podatka\textgreater predstavlja tip elemenata niza, a za
\textless tip\(\_\)indeksa\textgreater se navodi ili tip podataka kome pripadaju
indeksi niza ili ``*'', što označava bilo koji celobrojni izraz proizvoljne
veličine. Npr.

\begin{lstlisting}
int example_array1 [*]; // niz int-ova

// niz 8-bitnih vektora ciji
// su indeksi string-ovi
bit [7 : 0] example_array2 [string];

// niz logic elemenata koji
//su indeksirani sa example_class klasom
logic example_array3 [example_class];
\end{lstlisting}

SystemVerilog pruža veliki broj metoda za rad sa asocijativnim nizovima. One su:

\begin{itemize}
\item \emph{num()} – vraća broj elemenata u nizu
\item \emph{delete(index)} – briše element sa specificiran datim indeksom
\item \emph{exists(index)} – vraća 1 ukoliko postoji element na datom indeksu, 0
  u suprotnom
\item \emph{first(var)} – pridružuje vrednost prvog indeksa promenljivoj
  \emph{var} i vraća 0 ukoliko je niz prazan, 1 u suprotnom
\item \emph{last(var)} – pridružuje vrednost poslednjeg indeksa promenljivoj
  \emph{var} i vraća 0 ukoliko je niz prazan, 1 u suprotnom
\item \emph{next(var)} – ukoliko postoji, pridružuje vrednost sledećeg indeksa
  promenljivoj \emph{var} i vraća 1, u suprotnom \emph{var} ostaje nepromenjena
  i vraća se 0
\item \emph{prev(var)} – ukoliko postoji, pridružuje vrednost prethodnog indeksa
  promenljivoj \emph{var} i vraća 1, u suprotnom \emph{var} ostaje nepromenjena
  i vraća se 0
\end{itemize}

U nastavku je dat primer upotrebe asocijativnih nizova. Uočiti zašto se
iteracija kroz sve elemente niza vrši sa \emph{foreach}, a ne \emph{for}
petljom. Alternativno, moguće je iterirati kroz sve elemente koristeći metode
\emph{next()} ili \emph{prev()} u petlji.

\lstinputlisting[caption=Asocijativni nizovi, label=lst:assoc_array]{code/v2_assoc_array.sv}

%----------------------------------------------------------------------------------------

\subsection{Redovi}

Redovi (engl. \emph{queues}) u SystemVerilog-u predstavljaju kombinaciju
dinamičkih nizova i povezanih lista (engl. \emph{linked lists}).
Kao i kod dinamičkih nizova, bilo kom elementu se može pristupati preko indeksa
(bez troškova prolaska kroz sve prethodne elemente kao kod liste), ali se i, kao
kod povezanih lista mogu dodavati ili izbacivati elementi bilo gde u redu (bez
troškova kopiranja i kreiranja nove strukture kao kod dinamičkog niza).
Redovi se deklarišu kao \emph{unpacked} nizovi, ali se navodi znak ``\(\$\)''
kao veličina, odnosno:

\begin{lstlisting}
<data_type> <queue_name> [$];
\end{lstlisting}
% $ dumb comment to get colors right :)

Što se tiče indeksiranja, ``0'' predstavlja prvi element, a ``\(\$\)''
poslednji.
Vitičaste zagrade se koriste za inicijalizaciju redova, ali i za dodavanje ili
brisanje elemenata.
Operatori sa rad sa redovima su slični operatorima \emph{unpacked} nizova.\\

Postoje i ugrađene metode za rad sa redovima:

\begin{itemize}
\item \emph{size()} – vraća broj elemenata u redu
\item \emph{insert(indeks, vrednost)} - ubacuje element na mesto sa datim
  indeksom
\item \emph{delete(indeks)} - briše element sa datim indeksom. Ako indeks nije
  zadat, briše se ceo red
\item \emph{push\(\_\)front(vrednost)} - ubacuje element na početak reda
\item \emph{push\(\_\)back(vrednost)} - ubacuje element na kraj reda
\item \emph{pop\(\_\)front()} - briše i vraća element sa početka reda
\item \emph{pop\(\_\)back()} - briše i vraća element sa kraja reda
\end{itemize}

U nastavku su dati primeri korićenja redova i izvršeno je poređenje upotrebe
operatora sa upotrebom metoda. Naredni kodni fragmenti su su ekvivalentni.

\lstinputlisting[caption=Redovi - operatori, label=lst:queue_operators]{code/v2_queue_operators.sv}
\lstinputlisting[caption=Redovi - metode, label=lst:queue_methods]{code/v2_queue_methods.sv}

Treba napomenuti da iako postoje povezane liste u SystemVerilog-u, njihova
upotreba se ne preporučuje zato što su redovi mnogo efikasniji i lakši za
korišćenje, pa se ovde neće detaljnije obrađivati.

%----------------------------------------------------------------------------------------

\subsection{Metode}

U SystemVerilog-u postoji mnoštvo metoda za rad sa nizovima i redovima.
Ove metode se mogu koristiti na bilo kom \emph{unpacked} nizu, odnosno na
nizovima fiksne veličine, dinamičkim i asocijativnim nizovima i redovima.
Sintaksa je: 

\begin{verbatim}
array_method_call ::=
    expression.array_method_name { attribute_instance }
      | [ ( list_of_arguments ) ] [ with ( expression ) ]
\end{verbatim}

\emph{with} klauza je opciona i prihvata izraz u zagradama.\\

U nastavku je dat pregled ovih metoda.

\begin{itemize}
\item Metode lokacije (povratna vrednost je red):
  \begin{itemize}
  \item \emph{find()} - vraća sve elemente koji zadovoljavaju dati izraz
  \item \emph{find\(\_\)index()} - vraća indekse svih elemenata koji
    zadovoljavaju dati izraz
  \item \emph{find\(\_\)first()} - vraća prvi element koji zadovoljava dati
    izraz
  \item \emph{find\(\_\)first\(\_\)index()} - vraća indeks prvog elementa koji zadovoljava
    dati izraz
  \item \emph{find\(\_\)last()} - vraća poslednji element koji zadovoljava dati izraz
  \item \emph{find\(\_\)last\(\_\)index()} - vraća indeks poslednjeg elementa koji
    zadovoljava dati izraz
  \item \emph{min()} - vraća element sa minimalnom vredošću
  \item \emph{max()} - vraća element sa maksimalnom vrednošću
  \item \emph{unique()} - vraća sve elemente čija je vrednost jedinstvena
  \item \emph{unique\(\_\)index()} - vraća indekse svih elemenata čija je vrednost
    jedinstvena
  \end{itemize}
\item Metode uređivanja:
  \begin{itemize}
  \item \emph{reverse()} - preokreće sve elemente
  \item \emph{sort()} - sortira elemente u rastućem redosledu
  \item \emph{rsort()} - sortira elemente u opadajućem redosledu
  \item \emph{shuffle()} - randomizuje redosled elemenata
  \end{itemize}
\item Metode redukcije:
  \begin{itemize}
  \item \emph{sum()} - vraća sumu svih elemenata
  \item \emph{product()} - vraća proizvod svih elemenata
  \item \emph{and()} - vraća \emph{bitwise} \emph{and} od svih elemenata
  \item \emph{or()} - vraća \emph{bitwise} \emph{or} od svih elemenata
  \item \emph{xor()} - vraća \emph{bitwise} \emph{xor} od svih elemenata
  \end{itemize}
\end{itemize}

Za pojedine metode je \emph{with} klauza opciona, za pojedine obavezna, a za
neke nedozvoljena.
Sve \emph{find} metode moraju sadržati i \emph{with} klauzu, dok je za
\emph{min}, \emph{max} i \emph{unique} metode ona opciona.
Za \emph{sort}, \emph{rsort} i sve metode redukcije je takođe opciona.
Ali je za metode \emph{reverse} i \emph{shuffle} nedozvoljena.
U nastavku je dato nekoliko primera:

\lstinputlisting[caption=Primer upotrebe metoda, label=lst:array_methods]{code/v2_array_methods.sv}

\subsection{Zadaci}

\paragraph{Zadatak}

Kod \ref{lst:queue_examples} prikazuje upotrebu redova u SystemVerilog-u (fajl
``queue\(\_\)examples.sv'' u pratećim materijalima).
Proučiti način upotrebe redova, korišćenje operatora i metoda. 
Koji će biti rezultat izvršavanja datog koda?

\lstinputlisting[caption=Redovi - zadatak, label=lst:queue_examples]{code/v2_queue_examples.sv}


%========================================================================================
% Section
%========================================================================================

\section{Kastovanje}

Zbog velikog broja tipova podataka, ponekad je potrebno dodeliti promenljivu
jednog tipa promenljivoj drugog tipa odnosno uraditi \emph{cast}. Postoje dve
vrste \emph{cast}-ovanja: statički i dinamički.

%----------------------------------------------------------------------------------------

\subsection{Statički kast}

Tip podatka se može promeniti korišćenjem \emph{cast} (``\textquotesingle\'') operacije.
Potrebno je navesti željeni tip i izraz koji želimo da promenimo. Npr.

\begin{lstlisting}
int'(2.5 + 2.3); // realni broj postaje int
\end{lstlisting}

%----------------------------------------------------------------------------------------

\subsection{Dinamički kast}

SystemVerilog sadrži sistemski task \emph{\(\$\)cast} koji se koristi za
dinamičko \emph{cast}-ovanje.
Dinamički \emph{cast} omogućava proveru \emph{out-of-bounds} vrednosti.
\emph{\(\$\)cast} se može pozvati na dva načina: kao funkcija ili kao task.
Sintaksa poziva je:

\begin{lstlisting}
function int $cast( singular dest_var, singular source_exp );
task $cast( singular dest_var, singular source_exp );
\end{lstlisting}

Razlika je u reakciji na pogrešne dodele.
Ako se poziva kao task, \emph{\(\$\)cast} pokušava da dodeli \emph{source} izraz
\emph{destination} promenljivoj i ukoliko je dodela neuspešna javlja se
\emph{runtime error}, a promenljiva ostaje nepromenjena.
Ako se poziva kao funkcija, \emph{\(\$\)cast} će u neuspelom slučaju vratiti 0
da signalizira grešku, odnosno vratiće 1 pri uspešnoj operaciji.\\

Bitno je uočiti razliku između statičnog i dinamičkog \emph{cast}-a.
\emph{\(\$\)cast} vrši proveru u vreme izvršavanja (engl. \emph{at runtime}).
Provera tipa se ne vrši od strane kompajlera.
Što nije slučaj kod statičnog \emph{cast}-a.
Razlike i način odabira odgovarajuće vrste se najbolje uočava na primeru
koristeći \emph{enum}:

\lstinputlisting[caption=Kastovanje, label=lst:cast]{code/v2_cast.sv}

%========================================================================================
% Section
%========================================================================================

\section{Objektno orijentisano programiranje}

Objektno orijentisano programiranje omogućava kreiranje kompleksnih tipova
podataka i grupisanje istih sa metodama. U pogledu verifikacije, olakšava
kreiranje testbenčeva i modela na višem nivou apstrakcije čime se postiže veća
produktivnost, lakše održavanje koda i laka ponovna upotreba. Objekto
orijentisani aspekti SysemVerilog-a veoma liče na osobine C++ jezika. U ovom
poglavlju je dat pregled osnovnih pojmova i dati primeri upotreba klasa u
SystemVerilog jeziku.

%----------------------------------------------------------------------------------------

\subsection{Osnovni pojmovi}

Klasa je tip koji sadrži podatke i rutine za manipulisanje tim podacima. U
nastavku je dat primer klase u SystemVerilog-u. Klasa \emph{transaction} sadrži
podatke o podacima i adresi i ima metodu za prikaz. Labele nakon definicija
funkcija ili klasa, odnosno bilo kog bloka, su opcione ali korisne jer čine kod
mnogo čitljivijim.

\lstinputlisting[caption=Primer klase, label=lst:transacton]{code/v2_transaction.sv}

Terminologija:

\begin{itemize}
\item \emph{Class} - klasa je bazični blok koji sadrži podatke i procedure
\item \emph{Object} - objekat je instanca klase
\item \emph{Handle} - pokazivač na objekat
\item \emph{Property} - polje (promenljiva) u klasi (npr. \emph{addr} i
  \emph{data} iz prethodnog primera)
\item \emph{Method} - metoda je proceduralni kod koji manipuliše promenljivama
  smešten u funkciju ili task (npr. \emph{display\(\_\)transaction} iz prethodnog
  primera)
\end{itemize}

%----------------------------------------------------------------------------------------

\subsubsection{Objekti}

Klasa definiše tip podatka, a objekat je instanca te klase. Za razliku od
modula, objekti su dinamični. Mogu se kreirati, uništavati, pristupati im se
bilo kada u toku izvršavanja koda. Objekat se koristi tako što se prvo deklariše
promenljiva odgovarajućeg tipa koja čuva pokazivač na taj tip, a zatim se kreira
objekat pozivanjem funkcije \emph{new} i dodeljuje toj promenljivoj. Npr.:

\begin{lstlisting}
transaction tr;
tr = new;
\end{lstlisting}

Objektima u SystemVerilog-u se uvek pristupa preko pokazivača. Međutim postoji
dosta razlika u odnosu na pokazivače u C-u, odnosno mnogo više ograničenja
(jedan od razloga za korišćenje reči \emph{handle} umesto \emph{pointer}). C
\emph{pointer}-i dozvoljavaju aritmetičke operacije, mogu se definisati za
proizvoljne tipove podataka, može im se dodeljivati adresa dok su sve ove
operacije zabranjene za SystemVerilog \emph{handle}. Takođe, kastovanje je
dosta ograničeno za razliku od C-a.\\

Važno je zapamtiti da se objekat kreira jedino pozivom funkcije \emph{new}. U
narednom primeru \emph{tr1} i \emph{tr2} pokazuju na isti objekat u memoriji i
bilo koje promene na \emph{tr1} će uticati na \emph{tr2} i obrnuto.

\begin{lstlisting}
transaction tr1, tr2;
tr1 = new;
tr2 = tr1;
tr2.addr = 3;
tr1.addr = 1;
$display(tr2.addr); // prikazuje vrednost 1
\end{lstlisting}
% $ another useless comment

Promenljivima unutar objekata i metodama se pristupa pomoću ``.''. Npr.

\begin{lstlisting}
tr.data_i = 4;
tr.display_transaction();
\end{lstlisting}

%----------------------------------------------------------------------------------------

\subsubsection{Konstruktor}

Kada se objekat kreira, poziva se funkcija \emph{new()} asoricana sa datom
klasom. Ova funkcija se zove konstruktor i ne sme specificirati povratni tip.
Svaka klasa ima podražumevani konstruktor koji alocira memoriju za dati objkat
(slično kao \emph{malloc} u C-u) i inicijalizuje promenljive. Promenljive koje
su tipa sa 2 stanja dobijaju vrednost 0, a one sa 4 stanja vrednost X. Naravno,
moguće je i definisati sopstveni konstruktor, umesto korišćenja podrazumevanog,
koji će odraditi željene operacije. Moguće je i prosleđivati vrednost
konstruktorima kako bi se omogućila veća kontrola tokom izvršavanja, npr.
konstruktor u klasi \emph{transaction} može biti:

\begin{lstlisting}
function new(logic [7:0] data_value = 8hAA,logic [2:0] addr_value = 0);
  data_i = data_value;
  addr = addr_value;
endfunction : new
\end{lstlisting}

Za razliku od C++ jezika, u SystemVerilog-u može postojati samo jedan
konstruktor. Takođe, neinicijalizovan handle ima specijalnu vrednost
\emph{null}.

%----------------------------------------------------------------------------------------

\subsubsection{\emph{Static}}

Kao i u C++-u, i SystemVerilog podržava statična polja i metode. Statična polja
se koriste kada je potrebno da neki član ima istu vrednost u svim instancama.
One su zajedničke za sve objekte date klase odnosno vrednost polja je ista u
svim objektima. Primer upotrebe bio bi npr. polje koje broji instance date klase
odnosno aktuelne objekte u programu. Statične metode mogu pristupati samo
statičnim poljima i metodama u klasi (pokušaj pristupa ne-statičnim poljima
rezultuje kompilacionom greškom). Takođe, statične metode ne mogu biti
virtualne (objašnjeno u narednim vežbama).\\

Statična polja i metode se deklarišu koristeći ključnu reč \emph{static}, npr.:

\begin{lstlisting}
class staticClassExample;
  static int count;
  int local_count;

  static function void increaseCount();
    count++;
    // local_count++; nije dozvoljeno
    $display("Count value = %0d\n", count);
  endfunction : increaseCount
    
endclass : staticClassExample
\end{lstlisting}
% $ this is getting annoying...

%----------------------------------------------------------------------------------------

\subsubsection{\emph{This}}

Ključna reč \emph{this} se koristi da bi se nedvosmisleno specificiralo polje
ili metoda trenutne instance klase. Korišćenje ove ključne reči je uglavnom
opciono i često izostavljeno, iako veoma doprinosi čitljivosti koda. Najčešće se
sreće u konstruktorima. Npr.:

\begin{lstlisting}
class thisExample;
  int x;

  function new(int x = 5);
    this.x = x;
    // this.x se odnosi na polje u klasi
    // x se odnosi na prosledjeni arugment
  endfunction : new

endclass : thisExample
\end{lstlisting}

%----------------------------------------------------------------------------------------

\subsubsection{\emph{Extern}}

Slično kao i u C++ jeziku, i u SystemVerilogu je moguće definisati metodu izvan
klase. Ovo se postiže korišćenjem ključne reči \emph{extern}. Za obimne metode
korisno je definisati ih izvan klase kako bi sam kod bio čitljiviji. Da bi se
metoda definisala izvan klase potrebno je u klasi navesti prototip metode kojem
prethodi ključna reč \emph{extern}, a prilikom same definicije metode dodati ime
klase i ``::'' (\emph{scope resolution operator}) ispred imena metode. Npr:

\begin{lstlisting}
class externExample;
  int x;
  extern task taskExample();
endclass : externExample

task externExample::taskExample();
  // do something
endtask : taskExample
\end{lstlisting}

Prototip metode u klasi i definiciji se moraju poklapati. Međutim, razni
simulatori različito reaguju na podrazumevane vrednosti argumente. Neki
dozvoljavaju navođenje podrazumevanih argumenata na oba mesta, a neki ne. Ali
pošto su podrazumevane vrednosti argumenata bitne samo za poziv metode, a ne i
za njenu implementaciju, neophodno ih je navesti samo prilikom navođenja
prototipa u klasi.

%----------------------------------------------------------------------------------------

\subsection{Enkapsulacija}

Enkapsulacija je tehnika skrivanja podataka unutar klase. Skriveni podaci su
dostupni samo u unutrašnjosti klase i može im se pristupati jedino preko metoda
klase. Enkapsulacija omogućava zaštitu članova klase jer je pristup dozvoljen
jedino ``poznatim'' korisnicima odnosno metodama čije je ponašanje poznato, čime
se može vršiti provera i ograničavanje vrednosti pojedinih članova. Način
enkapsulacije u SystemVerilog-u veoma liči na C++. Podrazumevani pristup je
\emph{public}. Ukoliko se eksplicitno ne navede pravo pristupa članu klase, on
je \emph{publilc} i tom članu (polju ili metodi) se može pristupati kako iz
unutrašnjosti, tako i iz spoljašnjosti klase, bez ikakvih ograničenja, koristeći
``.'' operator. Pored \emph{public} pristupa, postoje još dva prava pristupa:
\emph{local} i \emph{protected}. \emph{Local} pristup je nalik \emph{private}
pristupu u C++ jeziku. Članu klase koji je definisan kao \emph{local} se može
pristupati jedino iz unutrašnjosti klase (ne može im se pristupati ni iz klase
koja nasleđuje datu klasu). \emph{Protected} pristup se isto sreće u C++ jeziku
i omogućava pristup i u nasleđenim klasama, ali i dalje im se ne može
pristupati iz spoljašnjosti. Kontrola pristupa se vrši dodavanjem kjučnih reči
\emph{local} ili \emph{protected} ispred deklarisanja člana klase, npr:

\lstinputlisting[caption=Enkapsulacija, label=lst:encapsulation]{code/v2_encapsulation.sv}

SystemVerilog je jezik za verifikaciju i, za razliku od klasičnih programskih
jezika kao što je C++, korišćenje enkapsulacije u praksi je retko. Lak pristup i
jednostavna kontrola elementima u verifikacionom okruženju je često bitnija od
dugoročne stabilnosti softvera, čak je često i potrebno ubacivati pogrešne
vrednosti kako bi se detaljno verifikovao dizajn. Zbog toga, u SystemVerilog-u,
članovi klase najčešće imaju podrazumevani, \emph{public}, pristup.

%----------------------------------------------------------------------------------------

\subsection{Nasleđivanje}

Nasleđivanje je veza između klasa koja omogućava preuzimanje sadržaja iz jedne
klase (\emph{parent} klasa) i uz eventualne modifikacije i dodatke kreira
izvedenu (\emph{child}) klasu. \emph{Child} klasa nasleđuje sva polja i metode
iz parent klase, može ih modifikovati, ali može i dodavati nova polja i metode.
Npr. ukoliko uzmemo primer \emph{transaction} klase sa početka poglavlja, tokom
razvoja dizajna moguće je da se ukaže potreba za dodavanjem još jednog člana
koji čuva npr. \emph{en} ili \emph{rw}. Tada bi, umesto kreiranja nove klase,
nasledili \emph{transaction} klasu i dodali potreban sadržaj. Nasleđivanje se
realizuje korišćenjem ključne reči \emph{extends}.

\lstinputlisting[caption=Nasleđivanje, label=lst:new_transaction]{code/v2_new_transaction.sv}

%----------------------------------------------------------------------------------------

\subsubsection{Redefinisane metode}

Metode u \emph{child} klasi mogu biti nasleđene iz \emph{parent} klase bez
modifikacija, mogu biti redefinisane ili mogu biti potpuno nove. Podrazumevano
se sve metode iz \emph{parent} klase nasleđuju u \emph{child} klasi, a zatim se
iste mogu redefinisati ukoliko je to potrebno - jednostavno se navede nova
definicija metode koja će se koristiti u \emph{child} klasi umesto originalne.

%----------------------------------------------------------------------------------------

\subsubsection{\emph{Super}}

Ključna reč \emph{super} se koristi u \emph{child} klasi kako bi se
referencirali članovi \emph{parent} klase. Neophodno ju je koristiti jedino kada
je metoda redefinisana u \emph{child} klasi, a potrebno je pristupiti
originalnoj metodi u \emph{parent} klasi.\\

Pogledati primer redefinisanja metode \emph{display\(\_\)transaction} u
\emph{new\(\_\)transaction} klasi.

%----------------------------------------------------------------------------------------

\subsubsection{Konstruktor}

Kada se \emph{child} klasa instancira, poziva se kontruktor čija je prva akcija
da pozove konstruktor \emph{parent} klase čime se postiže pravilan redosled
pozivanja konstruktora od bazne klase do svih nasleđenih klasa. Ukoliko
konstruktor u \emph{parent} klasi ima argumente, konstruktor u \emph{child}
klasi mora postojati i mora pozivati \emph{parent} konstruktor na prvoj liniji
(\emph{super.new(...)}).

%========================================================================================
% Section
%========================================================================================

\section{Interfejs i \emph{package}}

Interfejsi i \emph{package}-i su dve često korišćene konstrukcije u
SystemVerilog-u. Sa porastom kompleksnosti okruženja postaju neophodni radi
čitljivosti i ponovne upotrebe koda, ali i olakšavanja same verifikacije. U ovom
poglavlju je dat kratak pregled, dok će se potrebne funkcionalnosti detaljno
obraditi u narednim vežbama.

%----------------------------------------------------------------------------------------

\subsection{Interfejsi}

Interfejsi služe za povezivanje dizajna i testbenča. Cilj je obuhvatiti svu
komunikaciju na jednom mestu. Svi signali koji povezuju DUT i testbenč treba da
se nalaze u interfesima, čime se postiže jasna struktura koda. Interfejsi
povećavaju fleksibilnost koda, olakšavaju pisanje protokol čekera i olakšavaju
upotrebu jer se interfesi mogu instancirati i koristiti kao promenljive. Kao i
moduli, i interfejsi mogu sadržati parametre, funkcije, taskove, promenljive,
... Definišu se ključnom reči \emph{interface} i mogu sadržati ulazne, izlazne i
ulazno-izlazne portove. Interfejsi se instanciraju kao i moduli i to samo
jednom, a zatim se pokazivač prosleđuje svim objektima koji treba da koriste
ovaj interfejs.

\lstinputlisting[caption=Interfejs, label=lst:interface]{code/v2_interface.sv}

Umesto da se pojedinačno kreiraju i instanciraju svi signali (\emph{addr},
\emph{data}, ...), oni se grupišu u interfejs koji se tada instancira u top modulu,
povezuje sa DUT-om i dalje prosleđuje svim objektima kojima je potreban pristup
(drajverima, monitorima, ...).

%----------------------------------------------------------------------------------------

\subsection{\emph{Package}}

Razvojem verifikacionog okruženja, raste i broj fajlova koji se koriste. Pošto
se, po konvenciji, svaka klasa definiše u posebnom fajlu, ubrzo može postati
nezgodno pratiti sve potrebne fajlove u projektu. Takođe, ukoliko postoje
funkcije koje se koriste na više mesta ili parametri ili osobine potrebne u više
klasa ili modula, korisno je definisati ih samo na jednom mestu. \emph{Package}
ovo omogućava.\\

Podacima u \emph{package}-u se može pristupati na dva načina: preko \emph{scope
  resolution operatora} (``::'') ili pomoću \emph{import} naredbe. U nastavku je
dat primer:

\lstinputlisting[caption=Package, label=lst:package]{code/v2_package.sv}

%========================================================================================
% Section
%========================================================================================

\section{Zadaci}

\paragraph{Zadatak}

U nastavku je dat primer testbenča koji se koristi za verifikaciju memorije.
Sastoji se od nekoliko klasa i modula.
Svi fajlovi se nalaze u pratećim materijalima za ovu vežbu.

\begin{itemize}
\item kod \ref{lst:tr} - primeri \emph{transaction} klase
\item kod \ref{lst:driver} - primer drajver klase koja generiše signale
\item kod \ref{lst:mem_pkg} - \emph{package}; include svih fajlova na jednom mestu
\item kod \ref{lst:mem_if} - interfejs; služi za povezivanje DUT-a i okruženja
\item kod \ref{lst:mem} - jednostavan primer memorije koji služi kao DUT
\item kod \ref{lst:top} - glavni modul; u njemu se vrši povezivanje DUT-a i
  okruženja, generišu se \emph{clk} i \emph{reset} signali i startuje drajver
\item kod \ref{lst:run} - skripta za pokretanje simulacije
\end{itemize}

Analizirati sve fajlove, uočiti način povezivanja, primer nasleđivanja u
\emph{transaction} klasi, kao i upotrebu jedne klase u drugoj
(\emph{transaction} u drajver klasi).

\lstinputlisting[caption=Transackije, label=lst:tr]{code/v2_tr.sv}
\lstinputlisting[caption=Drajver, label=lst:driver]{code/v2_driver.sv}
\lstinputlisting[caption=Package, label=lst:mem_pkg]{code/v2_memory_pkg.sv}
\lstinputlisting[caption=Interfejs, label=lst:mem_if]{code/v2_memory_if.sv}
\lstinputlisting[caption=Memorija, label=lst:mem]{code/v2_memory.sv}
\lstinputlisting[caption=Top, label=lst:top]{code/v2_top.sv}
\lstinputlisting[caption=Run, label=lst:run]{code/v2_run.do}

\paragraph{Zadatak}

Napisati klasu monitor koja nadgleda signale sa interfejsa. Svaku ispravnu
transackiju koju uoči (\emph{en} visok) treba da sačuva u redu i da na kraju ispiše
sve uočene transakcije.

%========================================================================================
